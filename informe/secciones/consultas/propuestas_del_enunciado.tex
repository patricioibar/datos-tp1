\subsection{Consultas Propuestas por el Enunciado}

\subsubsection{¿Cuál es el estado que más descuentos tiene en total? ¿y en promedio?}

Para esta consulta se tomaron algunas consideraciones:
\begin{itemize}
    \item Se consideraron los 50 Estados de los Estados Unidos. No se consideraron territorios ni estados militares.
    \item El estado a considerar es el encontrado en la columna \texttt{billing\_address} de la tabla \texttt{orders}.
    \item Se consideraron los "estados con más descuentos" a aquellos que poseen la mayor cantidad de órdenes con descuentos aplicados. Se considera que una orden tiene un descuento si el monto del descuento no es nulo y es mayor a 0.
    \item La segunda parte de la consulta (``¿y en promedio?'') se interpreta como el estado con mayor promedio en el valor de \texttt{discount\_amount}.
\end{itemize}

Para poder realizar esta consulta, fue necesario extraer el estado y el código postal de la columna \texttt{billing\_address} de la tabla \texttt{orders}. 

Las direcciones parecían seguir un patrón que me facilitó extraer el estado y el código postal mediante una expresión regular.
Para esto, primero se normalizaron los datos de la columna y luego se utilizó la siguiente expresión regular:

\begin{lstlisting}[language=Python]
orders["billing_address"] = orders["billing_address"].str.upper()
orders.fillna({"billing_address":"UNDEFINED"}, inplace=True)
pattern = r'([A-Z]{2})\s(\d{5})'
orders[["state", "zip_code"]] = orders["billing_address"].str.extract(pattern)
\end{lstlisting}

La expresión regular extrae dos grupos: el primero corresponde al estado (dos letras mayúsculas) y el segundo al código postal (cinco dígitos).
Para verificar que esta extracción fuera exitosa se realizaron las siguientes comprobaciones, para las cuales se obtuvieron resultados positivos (ver Anexo \ref{anexo:output_validacion_consulta1}).

\begin{lstlisting}[language=Python, xleftmargin=25pt, xrightmargin=25pt, ]
null_state_and_addr = orders["state"].isna() & orders["billing_address"].str.contains("UNDEFINED")

print("Todas las filas que tienen estado nulo, tienen direccion de facturacion indefinida?", 
        "Si" if null_state_and_addr.sum() == orders["state"].isna().sum() else "No")
\end{lstlisting}

Finalmente, para responder la consulta, se creó un filtro que deja fuera los estados militares y territorios llamado \texttt{not\_states\_filter} (ver Anexo \ref{anexo:output_filtro_estados}). Se filtraron las órdenes conservando solo aquellas que tenían un monto de descuento mayor a 0 y que cumplían con el filtro de estados, y se agruparon por estado, contando la cantidad de órdenes con descuento por estado.

\begin{lstlisting}[language=Python, xleftmargin=35pt, xrightmargin=35pt, ]
orders_with_discount = orders.loc[orders["discount_amount"] > 0]
                                .loc[not_states_filter]

quantity_of_orders_with_discounts_by_state = orders_with_discount.groupby("state")["order_id"]
                                                                        .count().reset_index()
\end{lstlisting}

Luego, para visualizar mejor los resultados, se renombraron las columnas y se agregó una columna con el nombre del estado utilizando \texttt{map} (ver Anexo \ref{anexo:output_formateo_resultados}). Se visualizaron los 5 estados con más órdenes utilizando la función \texttt{nlargest} de pandas.

\begin{lstlisting}[language=Python, xleftmargin=35pt, xrightmargin=35pt]
print("\nTop 5 estados con mas ordenes con descuentos:")
quantity_of_orders_with_discounts_by_state.nlargest(5, "Cantidad de Ordenes con Descuento")
\end{lstlisting}
\begin{table}[H]
\centering

\begin{tabular}{|c|c|c|}
\hline
\textbf{Código de Estado} & \textbf{Cantidad de Órdenes con Descuento} & \textbf{Nombre del Estado} \\
\hline
LA & 13,950 & Louisiana \\
MO & 13,940 & Missouri \\
IL & 13,930 & Illinois \\
KY & 13,903 & Kentucky \\
IA & 13,873 & Iowa \\
\hline
\end{tabular}
\caption{Top 5 estados con más órdenes con descuentos}
\end{table}

Entonces la respuesta a ``¿Cuál es el estado que más descuentos tiene en total?'' es Louisiana.

Luego, para encontrar el estado con mayor promedio en el valor de \texttt{discount\_amount}, se calculó el promedio del monto de descuento por estado y se utilizó nuevamente la función \texttt{nlargest} para obtener los 5 estados con mayor promedio.

\begin{lstlisting}[language=Python, xleftmargin=26pt, xrightmargin=26pt]
states_avg_discount = orders_with_discount.groupby("state")["discount_amount"].mean().reset_index()
states_avg_discount.nlargest(5, "discount_amount")
\end{lstlisting}

\begin{table}[H]
\centering
\begin{tabular}{|c|c|c|}
\hline
\textbf{Código de Estado} & \textbf{Promedio de Descuento (\$)} & \textbf{Nombre del Estado} \\
\hline
NC & 50.67 & North Carolina \\
GA & 50.46 & Georgia \\
OK & 50.42 & Oklahoma \\
CO & 50.34 & Colorado \\
MS & 50.32 & Mississippi \\
\hline
\end{tabular}
\caption{Top 5 estados con mayor promedio de monto de descuento}
\end{table}

Entonces la respuesta a ``¿y en promedio?'' es North Carolina.

\subsubsection{¿Cuáles son los 5 códigos postales más comunes para las órdenes con estado `Refunded'? ¿Y cuál es el nombre más frecuente entre los clientes de esas direcciones?}

\subsubsection{Para cada tipo de pago y segmento de cliente, devolver la suma y el promedio expresado como porcentaje, de clientes activos y de consentimiento de marketing.}

\subsubsection{Para los productos que contienen en su descripción la palabra `stuff', calcular el peso total de su inventario agrupado por marca}